\section{Matrices}


Las matrices son arreglos rectangulares de n\'umero que nos ayudan a codificar informaci\'on. Por ejemplo:
$$
\begin{pmatrix}
	a_{1,1} & a_{1,2} \\
	a_{2,1} & a_{2,2}
\end{pmatrix}
$$
puede ser \'util para codificar los coeficientes del sistema de ecuaciones:
$$
\begin{cases}
	a_{1,1}x+a_{1,2}y=b_{1}\\
	a_{2,1}x+a_{2,2}y=b_{2}
\end{cases}
$$



En general, una matriz tiene la forma 
\begin{equation}
	\label{A}
	\tag{A}
	\begin{pmatrix}
		a_{1,1} & a_{1,2} & \cdots & a_{1,n} \\
		\vdots & & & \vdots\\
		a_{m,1} & a_{m,2} & \cdots & a_{m,n} 
	\end{pmatrix}
\end{equation} 

Los sub\'indices de cada elemento $a_{i,j}$ denotan la posici\'on del mismo: $i$ es el n\'umero del \emph{rengl\'on} (contando de arriba a abajo), mientras que $j$ es el n\'umero de la columna (contanto de izquierda a derecha).



Podemos extraer renglones y columnas de la matrix \eqref{A}: El $i-$es\'imo rengl\'on es
$$
R_{i}=
\begin{pmatrix}
	a_{i,1} & \cdots & a_{i,n}
\end{pmatrix}
$$ 
mientras que la $j-$\'esima columna ser\'a
$$
C_{j}=
\begin{pmatrix}
	a_{j,1} \\
	\vdots \\
	a_{j,m}
\end{pmatrix}
$$



Diremos que la matriz \eqref{A} tiene dimensi\'on $m\times n.$ 

Si existe un conjunto de n\'umeros $F,$ tal que todos los elementos $a_{i,j}$ de la matriz pertenecen a dicho conjunto, diremos que la matriz tiene coeficientes en $F.$ 



\begin{rem}
	Para que las operaciones entre matrices est\'en bien definidas, es necesario que la suma, resta y multiplicaci\'on entre entre elementos de $F$ tambi\'en este bien definida. Por esto generalmente $F$ se elige como $\R$ o $\Z.$ 
\end{rem}



La colecci\'on de todas las matrices de dimensi\'on $m\times n$ con coeficientes en $F$ se denotar\'a por $$M_{m,n}(F).$$



\begin{defn}
	Las matrices de dimensi\'on $m\times 1$ se conocen como \emph{vectores columna,} mientras que las de dimensi\'on $1\times n$ se conocen como \emph{vectores rengl\'on.}
	
	
	La colecci\'on $M_{m,1}(F)$ de todos los vectores columna con coeficientes  comunmente se denota por $F^{m}.$  Mientras que la colecci\'on $M_{1,n}(F)$ de todos los vectores columna con coeficientes  comunmente se denota por $F^{n\ast}.$
	
\end{defn}


\subsection{Operaciones elementales}


Por brevedad, la matriz \eqref{A} se denota por $A=[a_{i,j}].$ 

En el caso de los vectores renglones y columnas, podemos omitir el sub\'indice fijo
$$R=[R_{1,j}]=[R_{j}], \; C=[C_{i,1}]=[C_{i}].$$



Si $B=[b_{i,j}]$ es otra matriz de dimensi\'on $m\times n,$ la suma se define como $$A+B=[a_{i.j}+b_{i,j}].$$ 

De manera similar, la resta se define como $$A-B=[a_{i,j}-b_{i,j}].$$



\begin{exmp}
	$$
	\begin{pmatrix}
		1 & -1 & 0 \\
		2 & 3 & -4
	\end{pmatrix}
	+
	\begin{pmatrix}
		7 & 0 & -1 \\
		2 & -1 & 5
	\end{pmatrix} =
	$$ 
	$$
	\begin{pmatrix}
		1 & -1 & 0 \\
		2 & 3 & -4
	\end{pmatrix}
	-
	\begin{pmatrix}
		7 & 0 & -1 \\
		2 & -1 & 5
	\end{pmatrix} =
	$$
\end{exmp}




Observe que para que la \emph{suma y resta} tenga sentido, ambas matrices deben tener exactamente las \emph{mismas dimensiones}. 

Despu\'es de ver la facilidad para definir la suma y resta, uno se ve tentado a definir la multiplicaci\'on de la misma forma. Pero tal definici\'on es poco \'util en las aplicaciones. 

Por esta raz\'on, desarrollaremos el concepto de multiplicaci\'on, a fin de poder aplicar esta operaci\'on en la resoluci\'on de Ejemplos.


\subsection{Multiplicaci\'on}

\begin{defn}
	Sean $R=[R_{j}]$ un vector rengl\'on y $C=[C_{i}]$ un vector columna, ambos de longitud $n.$
	
	El \emph{producto rengl\'on-columna} se define como
	\begin{equation}
		\label{RC}
		\tag{RC}
		RC=
		\begin{pmatrix}
			R_{1}& \cdots & R_{n}
		\end{pmatrix}
		\begin{pmatrix}
			C_{1} \\ \vdots \\ C_{n}
		\end{pmatrix}
		=
		\sum_{i=1}^{n} R_{j}C_{i}.
	\end{equation}
	
\end{defn}




\begin{exmp}
	Considere
	$$
	R=
	\begin{pmatrix}
		1 & 0 & -1
	\end{pmatrix}, \;
	C=
	\begin{pmatrix}
		2\\ 1 \\ -2
	\end{pmatrix}.
	$$
	
	Calcule $RC.$
	
\end{exmp}




\begin{exmp}
	Reescriba la siguiente ecuaci\'on, utilizando el \emph{producto rengl\'on-columna}:
	$$2x-3y+z=0.$$
\end{exmp}




\begin{defn}
	Sea $A=[a_{i,j}]\in M_{m\times n}$ y $B=[b_{j,k}]\in M_{n\times l}.$ Definimos su producto como
	\begin{equation}
		\label{AB}
		\tag{AB}
		AB=
		\begin{pmatrix}
			R_{i}C_{k}
		\end{pmatrix}
	\end{equation}
	donde $R_{i}$ es el $i-$\'esimo rengl\'on de $A$ y $C_{k}$ es la $k-$\'esima columna de $B.$
\end{defn}




\begin{rem}
	\begin{itemize}
		\item Para que esta multiplicaci\'on tenga sentido, los renglones de $A$ y las columnas de $B$ deber\'an tener la misma longitud $n.$
		
		\item La matriz resultante tendr\'a dimensi\'on $m \times l.$  
		\item A menos que $m=l,$ el producto $BA$ podr\'ia no estar definido. 
		\item Aun cuando $BA$ estuviera bien definido, el producto de matrices no es \emph{conmutativo,} es decir, generalmente tendremos que $$AB \neq BA.$$
	\end{itemize}
\end{rem}




\begin{exmp}
	Encuentre el producto $AB$ de las siguientes matrices
	$$A= \left(\begin{array}{r}
		0
	\end{array}\right) $$
	$$B= \left(\begin{array}{rr}
		0 & -1
	\end{array}\right) $$
	Solución:
	$$AB= \left(\begin{array}{rr}
		0 & 0
	\end{array}\right) $$
\end{exmp}



\begin{exmp}
	Encuentre el producto $AB$ de las siguientes matrices
	$$A= \left(\begin{array}{rr}
		0 & -1 \\
		-1 & 0 \\
		0 & 0
	\end{array}\right) $$
	$$B= \left(\begin{array}{rrr}
		0 & -1 & 0 \\
		0 & 0 & 0
	\end{array}\right) $$
	Solución:
	$$AB= \left(\begin{array}{rrr}
		0 & 0 & 0 \\
		0 & 1 & 0 \\
		0 & 0 & 0
	\end{array}\right) $$
\end{exmp}



\begin{exmp}
	Encuentre el producto $AB$ de las siguientes matrices
	$$A= \left(\begin{array}{r}
		-1 \\
		-1 \\
		0
	\end{array}\right) $$
	$$B= \left(\begin{array}{rrr}
		-1 & 0 & 0
	\end{array}\right) $$
	Solución:
	$$AB= \left(\begin{array}{rrr}
		1 & 0 & 0 \\
		1 & 0 & 0 \\
		0 & 0 & 0
	\end{array}\right) $$
\end{exmp}



\begin{exmp}
	Encuentre el producto $AB$ de las siguientes matrices
	$$A= \left(\begin{array}{r}
		6 \\
		-9 \\
		-10
	\end{array}\right) $$
	$$B= \left(\begin{array}{r}
		-5
	\end{array}\right) $$
	Solución:
	$$AB= \left(\begin{array}{r}
		-30 \\
		45 \\
		50
	\end{array}\right) $$
\end{exmp}



\begin{exmp}
	Encuentre el producto $AB$ de las siguientes matrices
	$$A= \left(\begin{array}{r}
		2
	\end{array}\right) $$
	$$B= \left(\begin{array}{rrr}
		-1 & 1 & -3
	\end{array}\right) $$
	Solución:
	$$AB= \left(\begin{array}{rrr}
		-2 & 2 & -6
	\end{array}\right) $$
\end{exmp}



\begin{exmp}
	Encuentre el producto $AB$ de las siguientes matrices
	$$A= \left(\begin{array}{rr}
		-1 & -3 \\
		-7 & -1
	\end{array}\right) $$
	$$B= \left(\begin{array}{r}
		-7 \\
		-4
	\end{array}\right) $$
	Solución:
	$$AB= \left(\begin{array}{r}
		19 \\
		53
	\end{array}\right) $$
\end{exmp}



\begin{exmp}
	Rescriba el siguiente sistema de ecuaci\'on en forma matricial y encuentre su soluci\'on:
	$$
	\begin{cases}
		-x-3y=19\\
		-7x-y=53
	\end{cases}
	$$
\end{exmp}


